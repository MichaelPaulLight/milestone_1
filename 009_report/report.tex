% Options for packages loaded elsewhere
\PassOptionsToPackage{unicode}{hyperref}
\PassOptionsToPackage{hyphens}{url}
\PassOptionsToPackage{dvipsnames,svgnames,x11names}{xcolor}
%
\documentclass[
  11pt,
  letterpaper,
  DIV=11,
  numbers=noendperiod]{scrartcl}

\usepackage{amsmath,amssymb}
\usepackage{setspace}
\usepackage{iftex}
\ifPDFTeX
  \usepackage[T1]{fontenc}
  \usepackage[utf8]{inputenc}
  \usepackage{textcomp} % provide euro and other symbols
\else % if luatex or xetex
  \usepackage{unicode-math}
  \defaultfontfeatures{Scale=MatchLowercase}
  \defaultfontfeatures[\rmfamily]{Ligatures=TeX,Scale=1}
\fi
\usepackage{lmodern}
\ifPDFTeX\else  
    % xetex/luatex font selection
\fi
% Use upquote if available, for straight quotes in verbatim environments
\IfFileExists{upquote.sty}{\usepackage{upquote}}{}
\IfFileExists{microtype.sty}{% use microtype if available
  \usepackage[]{microtype}
  \UseMicrotypeSet[protrusion]{basicmath} % disable protrusion for tt fonts
}{}
\makeatletter
\@ifundefined{KOMAClassName}{% if non-KOMA class
  \IfFileExists{parskip.sty}{%
    \usepackage{parskip}
  }{% else
    \setlength{\parindent}{0pt}
    \setlength{\parskip}{6pt plus 2pt minus 1pt}}
}{% if KOMA class
  \KOMAoptions{parskip=half}}
\makeatother
\usepackage{xcolor}
\usepackage[top=30mm,left=20mm,right=20mm,bottom=30mm]{geometry}
\setlength{\emergencystretch}{3em} % prevent overfull lines
\setcounter{secnumdepth}{5}
% Make \paragraph and \subparagraph free-standing
\ifx\paragraph\undefined\else
  \let\oldparagraph\paragraph
  \renewcommand{\paragraph}[1]{\oldparagraph{#1}\mbox{}}
\fi
\ifx\subparagraph\undefined\else
  \let\oldsubparagraph\subparagraph
  \renewcommand{\subparagraph}[1]{\oldsubparagraph{#1}\mbox{}}
\fi


\providecommand{\tightlist}{%
  \setlength{\itemsep}{0pt}\setlength{\parskip}{0pt}}\usepackage{longtable,booktabs,array}
\usepackage{calc} % for calculating minipage widths
% Correct order of tables after \paragraph or \subparagraph
\usepackage{etoolbox}
\makeatletter
\patchcmd\longtable{\par}{\if@noskipsec\mbox{}\fi\par}{}{}
\makeatother
% Allow footnotes in longtable head/foot
\IfFileExists{footnotehyper.sty}{\usepackage{footnotehyper}}{\usepackage{footnote}}
\makesavenoteenv{longtable}
\usepackage{graphicx}
\makeatletter
\def\maxwidth{\ifdim\Gin@nat@width>\linewidth\linewidth\else\Gin@nat@width\fi}
\def\maxheight{\ifdim\Gin@nat@height>\textheight\textheight\else\Gin@nat@height\fi}
\makeatother
% Scale images if necessary, so that they will not overflow the page
% margins by default, and it is still possible to overwrite the defaults
% using explicit options in \includegraphics[width, height, ...]{}
\setkeys{Gin}{width=\maxwidth,height=\maxheight,keepaspectratio}
% Set default figure placement to htbp
\makeatletter
\def\fps@figure{htbp}
\makeatother

\KOMAoption{captions}{tableheading}
\usepackage{fancyhdr}
\pagestyle{fancy}
\fancyhead[R]{\thepage}
\makeatletter
\@ifpackageloaded{caption}{}{\usepackage{caption}}
\AtBeginDocument{%
\ifdefined\contentsname
  \renewcommand*\contentsname{Table of contents}
\else
  \newcommand\contentsname{Table of contents}
\fi
\ifdefined\listfigurename
  \renewcommand*\listfigurename{List of Figures}
\else
  \newcommand\listfigurename{List of Figures}
\fi
\ifdefined\listtablename
  \renewcommand*\listtablename{List of Tables}
\else
  \newcommand\listtablename{List of Tables}
\fi
\ifdefined\figurename
  \renewcommand*\figurename{Figure}
\else
  \newcommand\figurename{Figure}
\fi
\ifdefined\tablename
  \renewcommand*\tablename{Table}
\else
  \newcommand\tablename{Table}
\fi
}
\@ifpackageloaded{float}{}{\usepackage{float}}
\floatstyle{ruled}
\@ifundefined{c@chapter}{\newfloat{codelisting}{h}{lop}}{\newfloat{codelisting}{h}{lop}[chapter]}
\floatname{codelisting}{Listing}
\newcommand*\listoflistings{\listof{codelisting}{List of Listings}}
\makeatother
\makeatletter
\makeatother
\makeatletter
\@ifpackageloaded{caption}{}{\usepackage{caption}}
\@ifpackageloaded{subcaption}{}{\usepackage{subcaption}}
\makeatother
\ifLuaTeX
  \usepackage{selnolig}  % disable illegal ligatures
\fi
\usepackage{bookmark}

\IfFileExists{xurl.sty}{\usepackage{xurl}}{} % add URL line breaks if available
\urlstyle{same} % disable monospaced font for URLs
\hypersetup{
  pdftitle={model-visualization},
  colorlinks=true,
  linkcolor={blue},
  filecolor={Maroon},
  citecolor={Blue},
  urlcolor={Blue},
  pdfcreator={LaTeX via pandoc}}

\title{model-visualization}
\author{}
\date{}

\begin{document}
\maketitle

\setstretch{1.15}
\section{Project Motivation}\label{project-motivation}

\subsection{Background on Dialysis and Proposition
29}\label{background-on-dialysis-and-proposition-29}

Dialysis is a lifesaving treatment that removes waste from blood, acting
as an artificial kidney for those with chronic kidney disease. In recent
years, California has seen multiple attempts to increase regulations on
dialysis facilities through ballot initiatives:

\begin{itemize}
\tightlist
\item
  Proposition 8 in 2018
\item
  Proposition 23 in 2020
\item
  Proposition 29 in 2022
\end{itemize}

Proposition 29, which failed to pass, aimed to establish increased
regulations for both staffing and operations for the roughly 600
dialysis facilities in California, an estimated \$3.5 billion industry.
A key provision of the proposition required the presence of a physician
or licensed practitioner during all treatment hours, potentially
increasing each facility's costs by several hundred thousand dollars
annually.

\subsection{Debate Surrounding Dialysis
Regulations}\label{debate-surrounding-dialysis-regulations}

\begin{itemize}
\tightlist
\item
  \textbf{Proponents} argue that increased regulations improve patient
  safety and quality of care.
\item
  \textbf{Opponents} contend that the increase in healthcare costs is
  unwarranted and would limit care coverage by overwhelming facilities
  with costs and potentially
  \href{https://www.latimes.com/california/story/2022-11-08/2022-california-election-prop-29-vote-dialysis-clinic-results}{forcing
  them to close}.
\end{itemize}

\subsection{Project Scope and
Significance}\label{project-scope-and-significance}

Our project explores trends in:

\begin{enumerate}
\def\labelenumi{\arabic{enumi}.}
\tightlist
\item
  Dialysis facility access
\item
  Quality of care
\item
  Ballot results in California
\end{enumerate}

Using publicly available data from:

\begin{itemize}
\tightlist
\item
  Center for Medicare and Medicaid Services
\item
  Aggregated election results from California's Secretary of State
\end{itemize}

We analyze associations between dialysis care and voting behaviors,
specifically those related to recent statewide ballot initiatives
designed to regulate California's multibillion-dollar dialysis industry.

\subsection{Relevance and Novelty}\label{relevance-and-novelty}

\begin{itemize}
\tightlist
\item
  This project uses a novel approach to study an area of public interest
  relevant not just to California but the entire country.
\item
  To our knowledge, this is the first project of its kind to explore the
  possible association between voting patterns and the quality of
  dialysis care received.
\item
  The majority of dialysis treatments in California
  \href{https://lao.ca.gov/BallotAnalysis/Proposition?number=29&year=2022}{are
  covered by Medicare}, making the Medicare and Medicaid datasets
  particularly relevant to our analysis.
\end{itemize}

\subsection{Broader Context}\label{broader-context}

Investigative reporters, patient advocacy groups, and labor
organizations have spent significant resources over the past decade to
raise public awareness of the dialysis industry and its need for
regulation. Our project contributes to this ongoing discussion by
providing data-driven insights into the relationship between dialysis
care quality and voting behavior.

\section{Project Research Questions}\label{project-research-questions}

Our research questions are divided into two categories: primary and
secondary. The primary question serves as the main focus of our
analysis, while secondary questions provide additional insights through
further investigation of the data.

\subsection{Primary Research Question}\label{primary-research-question}

\textbf{Is the quality of care of dialysis facilities correlated with
voting in favor of or against dialysis industry regulation?}

\subsubsection{Key Assumptions}\label{key-assumptions}

\begin{enumerate}
\def\labelenumi{\arabic{enumi}.}
\tightlist
\item
  The relationship between Quality of Care and Voting Behavior is not
  confounded.
\item
  A vote in favor of any of the three propositions (Prop 8 in 2018, Prop
  23 in 2020, Prop 29 in 2022) can be interpreted as support for
  dialysis industry regulation.
\end{enumerate}

\subsubsection{Quality of Care Metrics}\label{quality-of-care-metrics}

To test this relationship under the outlined assumptions, we approximate
the quality of care using the following metrics:

\begin{itemize}
\tightlist
\item
  Five-star rating
\item
  Patient experience rating
\item
  Facility mortality rate
\item
  Number of available dialysis stations
\item
  Staff rating
\item
  Hospital readmission categorization (Worse than Expected, As Expected,
  Better than Expected)
\item
  Profit/non-profit designation
\item
  Parent company affiliation/independence
\end{itemize}

\subsubsection{Facility Categorization}\label{facility-categorization}

Observations of Quality of Care metrics in our data are categorized by:

\begin{itemize}
\tightlist
\item
  Year
\item
  County
\item
  City
\item
  Profit/Non-profit designation
\item
  Parent company affiliation/independence
\end{itemize}

\subsection{Secondary Research
Questions}\label{secondary-research-questions}

\begin{enumerate}
\def\labelenumi{\arabic{enumi}.}
\tightlist
\item
  What is the geographic coverage of dialysis facilities in California?
\item
  Is there any correlation between organizational structure
  (chain-owned, profit vs.~non-profit) and the quality of care?
\item
  Is there any association between the parent company of dialysis
  facilities and the quality of care?
\end{enumerate}

\section{Data Sources}\label{data-sources}

\subsection{Primary Data Sources}\label{primary-data-sources}

\subsubsection{Center for Medicare and Medicaid Services (CMS) Quarterly
Dialysis Facility Compare
Dataset}\label{center-for-medicare-and-medicaid-services-cms-quarterly-dialysis-facility-compare-dataset}

\begin{itemize}
\tightlist
\item
  \textbf{Source Details:}

  \begin{itemize}
  \tightlist
  \item
    Type: .zip \& .xlsx
  \item
    Years: 2017-2024
  \item
    Combined Size: 5,684 rows, 173 columns
  \item
    Link:
    \href{https://data.cms.gov/provider-data/topics/dialysis-facilities}{CMS
    Dialysis Facilities Data}
  \end{itemize}
\item
  \textbf{Key features}:

  \begin{itemize}
  \tightlist
  \item
    Star ratings for dialysis facilities
  \item
    Patient experience metrics
  \item
    Quality of care metrics
  \end{itemize}
\item
  \textbf{Insights provided}:

  \begin{itemize}
  \tightlist
  \item
    Patient satisfaction
  \item
    Clinical outcomes
  \item
    Doctor-patient communication
  \item
    Hospitalization rates
  \item
    Treatment effectiveness
  \end{itemize}
\item
  \textbf{Rating calculation}:

  \begin{itemize}
  \tightlist
  \item
    Patient experience star rating derived from bi-annual patient
    surveys
  \item
    Facility ratings based on metrics including:

    \begin{itemize}
    \tightlist
    \item
      Unplanned hospital readmissions
    \item
      Total and expected transfusions
    \item
      Ratio of deaths to expected deaths
    \item
      Waste removal efficiency across patient types
    \end{itemize}
  \end{itemize}
\end{itemize}

\subsubsection{California Secretary of State's (SOS) Statement of Vote
for Ballot
Measures}\label{california-secretary-of-states-sos-statement-of-vote-for-ballot-measures}

\begin{itemize}
\tightlist
\item
  \textbf{Source Details:}

  \begin{itemize}
  \tightlist
  \item
    Type: .xlsx
  \item
    Years: 2018, 2020, 2022
  \item
    Combined Size: 7,122 rows, 7 columns
  \item
    Link:
    \href{https://www.sos.ca.gov/elections/prior-elections/statewide-election-results}{CA
    SOS Statewide Election Results}
  \end{itemize}
\item
  \textbf{Elections covered}: November 2022, 2020, and 2018
\item
  \textbf{Focus}: Propositions regarding dialysis clinic requirements
\item
  \textbf{Geographic levels}:

  \begin{itemize}
  \tightlist
  \item
    Counties
  \item
    Sub-counties:

    \begin{itemize}
    \tightlist
    \item
      Congressional districts
    \item
      State senate districts
    \item
      State assembly districts
    \item
      Cities
    \end{itemize}
  \end{itemize}
\end{itemize}

\subsection{Secondary Data Source}\label{secondary-data-source}

\subsubsection{California Health and Human Services Specialty Care
Clinic Complete Data
Set}\label{california-health-and-human-services-specialty-care-clinic-complete-data-set}

\begin{itemize}
\tightlist
\item
  \textbf{Source Details:}

  \begin{itemize}
  \tightlist
  \item
    Type: .xlsx
  \item
    Years: 2013-2023
  \item
    Combined Size: 6,605 rows, 143 columns
  \item
    Link:
    \href{https://data.chhs.ca.gov/dataset/specialty-care-clinic-complete-data-set}{CHHS
    Specialty Care Clinic Data}
  \end{itemize}
\item
  \textbf{Time range}: 2013 through 2023
\item
  \textbf{Purpose}: Supplement CMS dataset with additional geographic
  data
\item
  \textbf{Additional facility-level features}:

  \begin{itemize}
  \tightlist
  \item
    Senate district
  \item
    Congressional district
  \item
    Latitude and longitude
  \end{itemize}
\end{itemize}

\subsection{Data Integration and Analysis
Potential}\label{data-integration-and-analysis-potential}

\begin{itemize}
\tightlist
\item
  Multiple geographic levels allow for aggregation and analysis at
  various scales
\item
  Combination of clinical data (CMS) with voting data (SOS) enables
  exploration of potential correlations between care quality and voting
  behavior
\item
  Supplementary geographic data enhances spatial analysis capabilities
\end{itemize}

\section{Data Manipulation Methods}\label{data-manipulation-methods}

Our workflow was broken down into five stages:

\begin{enumerate}
\def\labelenumi{\arabic{enumi}.}
\tightlist
\item
  Data Collection
\item
  Data Preparation
\item
  Database Management
\item
  Exploratory Data Analysis
\item
  Statistical Analysis
\end{enumerate}

\subsection{Data Collection and
Preparation}\label{data-collection-and-preparation}

\subsubsection{CMS Dialysis Facility
Dataset}\label{cms-dialysis-facility-dataset}

\paragraph{Organization and Import}\label{organization-and-import}

\begin{itemize}
\tightlist
\item
  Dataset structure: .zip files (one per year), containing multiple
  Excel files
\item
  Focus: Excel files relevant to facility general information, ratings,
  and patient survey results
\item
  Import result: Two separate parquet files at the facility level

  \begin{enumerate}
  \def\labelenumi{\arabic{enumi}.}
  \tightlist
  \item
    Patient survey responses
  \item
    Facility ratings and measurements
  \end{enumerate}
\end{itemize}

\paragraph{Challenges and Solutions}\label{challenges-and-solutions}

\begin{enumerate}
\def\labelenumi{\arabic{enumi}.}
\tightlist
\item
  Inconsistent File Naming Conventions

  \begin{itemize}
  \tightlist
  \item
    Issue: 2021 files named differently (e.g., patient survey data file
    named `59mq-zhts')
  \item
    Solution: Created a list of exact file names for selection, rather
    than using pattern matching
  \end{itemize}
\item
  Missing Data

  \begin{itemize}
  \tightlist
  \item
    Expected missing data: Survey non-responses
  \item
    Unexpected missing data: Administrative errors (e.g., missing
    columns in recent ICHPS raw data files)
  \item
    Solution for specific cases: Simple imputation during analysis
    (e.g., substituting 2018 `nan' values with 2019 values at the
    facility level)
  \end{itemize}
\end{enumerate}

\subsubsection{SOS Ballot Data}\label{sos-ballot-data}

\paragraph{Import and Selection}\label{import-and-selection}

\begin{itemize}
\tightlist
\item
  Data imported via URL for each relevant proposition year (2018, 2020,
  2022)
\item
  Selected columns containing `Kidney' or `Dialysis' for analysis
\item
  Geographic column manipulation:

  \begin{itemize}
  \tightlist
  \item
    Renamed columns
  \item
    Backfilled rows to address multi-level index (sub-counties under
    counties)
  \end{itemize}
\item
  Final output: Single ballot data parquet file

  \begin{itemize}
  \tightlist
  \item
    Includes year column
  \item
    Count and sub-county vote counts for each Dialysis Requirements
    Initiative proposition
  \end{itemize}
\end{itemize}

\paragraph{Challenges and Solutions}\label{challenges-and-solutions-1}

\begin{itemize}
\tightlist
\item
  Inconsistent naming conventions across years

  \begin{itemize}
  \tightlist
  \item
    2020 and 2022: `County Supervisorial'
  \item
    2018: `Supervisorial District'
  \end{itemize}
\item
  Solution: Standardized naming across all years
\end{itemize}

\subsubsection{CHHS Specialty Care Clinic Complete Data
Set}\label{chhs-specialty-care-clinic-complete-data-set}

\paragraph{Import and Alignment}\label{import-and-alignment}

\begin{itemize}
\tightlist
\item
  Downloaded Excel files for 2013 through 2023 (one per year)
\item
  Main challenge: Aligning pre-2018 data with 2018-forward structure
\item
  Process:

  \begin{enumerate}
  \def\labelenumi{\arabic{enumi}.}
  \tightlist
  \item
    Separated data into two dataframes: 2013-2017 and 2018-2023
  \item
    Used CHHS mapping dictionary to rename 2013-2017 columns
  \item
    Ensured consistent data types across both dataframes
  \item
    Merged dataframes using outer join on common columns
  \item
    Dropped rows with missing FAC\_NO (facility data)
  \end{enumerate}
\end{itemize}

\subsection{Database Management}\label{database-management}

\subsubsection{Data Merging and
Standardization}\label{data-merging-and-standardization}

\begin{itemize}
\tightlist
\item
  Standardized data types and column names across all datasets
\item
  Merged datasets:

  \begin{enumerate}
  \def\labelenumi{\arabic{enumi}.}
  \tightlist
  \item
    CMS facility rating dataset with CMS patient survey dataset
  \item
    Filtered CHHS dataset (dialysis clinics only) with merged CMS data
  \item
    Reshaped CMS and CHHS data by geographic level
  \item
    Merged geographic-level data with SOS Ballot Measures dataset
  \end{enumerate}
\end{itemize}

\subsubsection{Final Output}\label{final-output}

\begin{itemize}
\tightlist
\item
  Two parquet files:

  \begin{enumerate}
  \def\labelenumi{\arabic{enumi}.}
  \tightlist
  \item
    Data aggregated at city level
  \item
    Data aggregated at assembly district level
  \end{enumerate}
\end{itemize}

\subsubsection{Custom Relational Database
System}\label{custom-relational-database-system}

\begin{itemize}
\item
  We developed a custom Python-based relational database system to
  centralize our datasets and facilitate efficient data access and
  analysis. Key features include:

  \begin{enumerate}
  \def\labelenumi{\arabic{enumi}.}
  \tightlist
  \item
    \textbf{Table and View Structure}: Distinct tables for datasets with
    multiple views for focused data access.
  \item
    \textbf{Dynamic View Creation and Merging}: On-the-fly creation of
    custom views and combination of multiple views for complex analysis.
  \item
    \textbf{Conditional Querying}: User-defined conditions for precise
    data retrieval and filtering.
  \item
    \textbf{Efficient Data Access}: Quick and reliable access across the
    entire database.
  \item
    \textbf{Code Quality}: Object-oriented design with consistent naming
    conventions for improved maintainability and adaptability.
  \end{enumerate}
\item
  For detailed functionality, refer to the included database demo
  (Milestone\_1/004\_data-processing-scripts/002\_clean-raw-data/database\_demo.ipynb).
\end{itemize}

\section{Analysis and Insights}\label{analysis-and-insights}

This project employed a Bayesian approach to investigate the
relationship between dialysis facility quality metrics and voting
patterns on dialysis-related propositions in California.

By integrating these datasets, we were able to explore possible
relationships between dialysis facility quality metrics and voting
outcomes on dialysis-related propositions, an analysis that would not
have been possible with either dataset alone.

To maximize the granularity of our analysis, we chose to focus on
city-level data, which provided more detailed vote counts compared to
assembly district level data. We encoded voting outcomes as the
percentage of ``Yes'' votes in favor of the propositions, allowing for a
nuanced examination of support for dialysis industry regulation across
different localities.

This novel approach enabled us to investigate how various factors
related to dialysis care quality might influence public opinion and
voting behavior on healthcare policy initiatives, potentially offering
valuable insights for policymakers, healthcare providers, and voters
alike.

\subsection{Analysis Steps}\label{analysis-steps}

\subsection{Data Preparation}\label{data-preparation}

Before modeling and analysis, we underwent several data preparation
steps. These were kept until the analysis stage in the interest of
transparency. Steps taken included

\begin{enumerate}
\def\labelenumi{\arabic{enumi}.}
\tightlist
\item
  Imputing missing 2018 values using 2019 data for select variables.
\item
  Converting datatypes.\\
\item
  Calculating vote percentages in favor of regulation for each
  facility's city.
\item
  Filtering data to include years 2018, 2020, and 2022.
\item
  Aggregating data at the facility level, summarizing vote outcomes and
  facility characteristics.
\item
  Removing rows with missing values to perform a complete case analysis.
\end{enumerate}

\subsubsection{Model Construction}\label{model-construction}

We constructed a Bayesian multilevel model using the brms package in R.
This model allowed us to account for the hierarchical nature of our data
(facilities nested within cities and counties) while examining the
relationship between facility quality metrics and voting outcomes.

\subsubsection{Posterior Predictive
Checks:}\label{posterior-predictive-checks}

We performed posterior predictive checks to assess model fit and explore
relationships between key variables and voting outcomes.

\subsubsection{Visualization of
Effects:}\label{visualization-of-effects}

We created visualizations to illustrate the effects of key variables on
voting outcomes, such as staff rating and mortality rate.

\subsection{Insights}\label{insights}

\subsubsection{Staff Rating Impact:}\label{staff-rating-impact}

Our analysis revealed a negative relationship between staff ratings and
the percentage of votes in favor of dialysis regulation. This suggests
that areas with lower-rated dialysis facility staff were more likely to
support increased regulation. The effect of staff rating varied across
counties, with some counties showing stronger negative relationships
than others.

\subsubsection{Mortality Rate
Influence:}\label{mortality-rate-influence}

We found a positive relationship between facility mortality rates and
support for regulation. As mortality rates increased, the predicted vote
percentage in favor of regulation also increased. This suggests that
voters in areas with higher mortality rates at dialysis facilities were
more likely to support increased oversight.

\subsubsection{Patient Experience
Rating:}\label{patient-experience-rating}

Interestingly, our analysis showed a positive relationship between
patient experience ratings and support for regulation. This
counterintuitive finding suggests that even in areas where patients
report better experiences, there is still support for increased
regulation.

\subsubsection{Five Star Rating and Stations per
Facility:}\label{five-star-rating-and-stations-per-facility}

The associations of these metrics on voting behavior were weaker
compared to staff ratings, mortality rates, and patient experience
ratings. The estimated effects suggested by the posterior predictive
checks were clustered around zero.

\subsubsection{Chain Organization
Effects:}\label{chain-organization-effects}

The posterior predictive check for chain organizations showed varying
levels of support for regulation across different dialysis chains,
indicating that organizational factors may play a role in shaping public
opinion or voting behavior.

\subsubsection{Facility Size
Considerations:}\label{facility-size-considerations}

The number of dialysis stations (a proxy for facility size) showed a
slight positive relationship with voting in favor of regulation,
suggesting that areas with larger facilities might be more supportive of
increased oversight.

\subsection{Challenges and
Limitations}\label{challenges-and-limitations}

\subsubsection{Data Granularity:}\label{data-granularity}

While we had facility-level data for quality metrics, aggregating voting
data at the city and county level creates a mismatch in granularity,
potentially obscuring some finer-grained relationships.

\subsubsection{Temporal Alignment:}\label{temporal-alignment}

Our analysis assumed that facility metrics from a given year directly
influenced voting in that year's proposition. However, there may be lag
effects or longer-term trends that our current model doesn't capture.

\subsubsection{Confounding Factors:}\label{confounding-factors}

Despite our efforts to control for various factors, there may be
unmeasured confounders influencing both facility quality and voting
patterns that our model doesn't account for.

\subsubsection{Causal Interpretation:}\label{causal-interpretation}

While our model reveals associations between facility metrics and voting
patterns, caution should be exercised in interpreting these
relationships as causal. Further research, possibly including
quasi-experimental designs, would be needed to establish causal links.

In conclusion, our analysis provides novel insights into the complex
relationship between dialysis facility quality and public support for
industry regulation. The varying effects across counties and
organizations highlight the importance of considering local contexts
when interpreting these results. While we've uncovered several
interesting patterns, the complexity of the issue and limitations in our
data suggest that further research is needed to fully understand these
relationships and their policy implications.

\subsection{Results}\label{results}

\subsubsection{County-Level Effects}\label{county-level-effects}

\begin{figure}[H]

{\centering \includegraphics{../005_data-analysis-scripts/001_modeling/model-analysis.qmd\#fig-staff-rating-plot}

}

\caption{County-Level Staff Rating Effects}

\end{figure}%



\end{document}
